\section{State-Of-The-Art}
In this section related work and literature on the topic is discussed.

\subsection{Related Work}
While there is some literature available on component-based architectures, I am not aware of works that researched
moving from an inheritance based approach to a component based approach on the same product. I think this is the first
work describing the actual process. I am also not aware of any case studies of \OS{} real-time strategy game
architectures and object representation.

\subsection{Literature}
A small number of papers and books on the topic is available:
\begin{enumerate}
    \item A Generic Framework for Game Development - \cite{Fh02ageneric}
    \item Component Based Game Development – A Solution to Escalating Costs and Expanding Deadlines? - \cite{springerlink:10.1007/978-3-540-73551-95}
    \item A Software Architecture for Games - \cite{Doherty_2003}
    \item A flexible and expandable architecture for computer games - \cite{Plummer_2004}
    \item Game engine architecture - \cite{Gregory.2009}
    \item Game Architecture and Design: A New Edition - \cite{Rollings.2003}
\end{enumerate}

I found \citet{Fh02ageneric} particularly interesting, as it describes a system based only on components in great
detail. Haller not only explains the use of components, but also gives a very detailed proposition on how to handle
inter-component communication using a message system, based on and extending the QT GUI Framework\footnote{QT website:
\url{http://qt.nokia.com/}}.

Components can be connected to each other using strictly typed in and outputs, making it possible to connect components
using an external (graphical) tool, easily usable by non-programmers.

Most of the remaining literature embraces the use of components in a way or another, most don't go into any detail on
how to implement things though. Also the definition of a component varies quite a bit: for this project a component is a
single piece of code that contains small chunks of reusable code, but for some of the provided papers the word component
is used in a more abstract way like "the AI component", the "rendering component". 

\cite{Rollings.2003} emphasizes the need for well thought-out interfaces between the components. So that for example a
route finding library/class can easily be exchanged for another one. Thus avoiding tight coupling between the high
level components where possible.


