% {{{ Research methods
\section{Research Methods and Questions}
In this section we present our research questions and methods used in this work. 

\subsection{Research Questions}
We work on a set of four main research questions:
\begin{itemize}
	\item RQ1: Which architecture is used to describe objects in-game?
	\item RQ2: How are new objects added to the game?
	\item RQ3: Can existing objects easily be modified?
	\item RQ4: Are tools available to help with adding/modifying objects?
\end{itemize}

\subsection{RQ1}
\textit{Which architecture is used to describe objects in-game?}

The goal of this question is to find out if the game uses an inheritance based approach, a component based approach or
some other design to describe objects in game. 

\subsection{RQ2}
\textit{How are new objects added to the game?}

With this question we want to find out if many changes have to be made to the code to add new objects. We also want
to know if the objects are data or code driven. If they are data driven, we research which technology is used. Our goal
is to find the easiest method of adding objects to the game.

\subsection{RQ3}
\textit{Can existing objects easily be modified?}

Our goal is to assess whether existing objects are easily alterable or code has to be changed to modify them. We
research what the possible implications of changing objects are.

\subsection{RQ4}
\textit{Are tools available to help with adding/modifying objects?}

As creating game content is usually done by non-programmers, we want to assess how easy it is for them to add content to
the game and if there are tools to support them in the process.

\subsection{Research method}
The first part of this work is four case studies in which we research four existing open source games. All games are
mainly real-time strategy games, so their implementations face similar problems and are comparable to some degree. We
will then use the gained knowledge to improve the handling of objects in \UH{} by designing and
implementing a system combining the best practices we found in the case studies. Literature research is not a big part
of this project, as there is almost none available on the topic of game architectures, besides from massive
multiplayer online role playing games.
% }}}
