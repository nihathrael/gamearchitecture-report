% {{{ Research methods
\section{Research Methods and Questions}
In this section I present my research questions and methods used in this work. 

\subsection{Research Questions}
I work on a set of four main research questions:
\begin{itemize}
	\item RQ1: Which architecture is used to describe objects in-game?
	\item RQ2: How are new objects added to the game?
	\item RQ3: Can existing objects easily be modified?
	\item RQ4: Are tools available to help with adding/modifying objects?
\end{itemize}

I came up with these questions because of the problems I had when working on \UH{}. Adding new objects to the game
became more difficult over the years and has proven to be a cumbersome job, which almost almost involes a programmer.
This is not optimal, as time in very limited on \OS{} projects. Idealy the game-designer could add an object to the game
without involing a programmer in the process, allowing for a better seperation of work.
For this reason it would be nice to understand how other \OS{} games handle this problem and if those solutions are
applicable to \UH{}.

\subsection{RQ1}
\textit{Which architecture is used to describe objects in-game?}

The goal of this question is to find out if the game uses an inheritance based approach, a component based approach or
some other design to describe objects in game. 

\subsection{RQ2}
\textit{How are new objects added to the game?}

With this question I want to find out if many changes have to be made to the code to add new objects. I also want
to know if the objects are data or code driven. If they are data driven, I research which technology is used. My goal
is to find the easiest method of adding objects to the game.

\subsection{RQ3}
\textit{Can existing objects easily be modified?}

My goal is to assess whether existing objects are easily alterable or code has to be changed to modify them. I
research what the possible implications of changing objects are.

\subsection{RQ4}
\textit{Are tools available to help with adding/modifying objects?}

As creating game content is usually done by non-programmers, I want to assess how easy it is for them to add content to
the game and if there are tools to support them in the process.

\subsection{Research method}
I base my research methods on the \textit{Engineering Method} (a refirenment of the \textit{Scientific Method}) as
proposed in \cite{Basili:1992:EPS:647362.725507}. In this method existing solutions are examined in order to propose
better solutions based on the gained knowledge. This process is repeated until no further improvement is possible.
This method is based on the assumption that comparable solutions for the problem exist, which can be studied. 

I do this by first conducting a case-study on four existing \OS{} games,
in order to improve the first of them: \UH{}. This mainly means analyzing the code-bases of these projects and
studying their documentation. All four projects are \OS{} strategy games, which makes them comparable to some degree.

With the gained knowledge I design a new solution to the problem for \UH{} and follow up by implementing a first version
of the proposed design and analyzing the results.
% }}}
