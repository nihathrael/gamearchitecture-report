% {{{=================== Introduction ======================


\section{Introduction}

\subsection{Motivation}
\UH{}\footnote{Unknown Horizons website: \url{http://www.unknown-horizons.org}} is an \OS{} real-time strategy game developed by a team of programmers, artists, game
designers and many more around the globe. The first revision was committed in late 2007\footnote{First commit to \UH{}:
\url{https://github.com/unknown-horizons/unknown-horizons/commit/53eec12fd8bb52ac1a6ccfdb097296c479499dfd}}.

As the project evolved the game's code architecture grew dynamically, without much planned structure or
designed architecture. This resulted in a very tight coupling between the different components inside the game, making
it difficult to add/change certain functionality in the game. This became clear when adding the boat builder building
a while back, which resulted in months of fixing introduced bugs.

\UH{} uses the outdated idea of making use of multiple inheritance to compose its in-game objects. Besides introducing
very tight coupling between the different classes the current approach also does not allow non programmers to add new
assets to the game. For an \OS{} project this is clearly not ideal, as user contributions would add great value to the
project and save valuable programming time.

The idea for this project is to research how this problem is solved in similar \OS{} games and to transfer the results to
the \UH{} source-code. 

The following games have been chosen to be researched:
\begin{itemize}
    \item \BOW{}\footnote{Battle of Wesnoth website: \url{http://www.wesnoth.org}}
    \item \AD{}\footnote{0 A.D. website: \url{http://wildfiregames.com/0ad/}}
    \item \GLEST{}\footnote{\GLEST{} website: \url{http://megaglest.org/}}
\end{itemize}

\subsection{Problem Statement}
Three main questions should be answered by this project:
\begin{itemize}
    \item Which architecture do \OS{} games similar to \UH{} use to model their in-game objects?
    \item Can users add objects without modifying the game's code and if yes -- how?
    \item Can the \UH{} code-base be ported to a component-based architecture?
\end{itemize}

\subsection{Project Context}
This project is conducted for the course \textit{TDT4570 - Game Technology Specialization Project}\footnote{TDT4570 Project
description: \url{http://www.idi.ntnu.no/emner/tdt4570/}} which is part of
\textbf{NTNU}'s computer science master program.

%}}}

