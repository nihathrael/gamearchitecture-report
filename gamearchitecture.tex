% Template for computer science thesis at the TU Munich
% Authors Benedikt Mas y Parareda, Johannes Becker, Gunnar Schroeder, Elmar Juergens

\documentclass[11pt,a4paper]{article}

%Enable DVI forward search
%\usepackage[active]{srcltx}

\usepackage{multicol}
\usepackage[utf8x]{inputenc}
\usepackage[T1]{fontenc}
\usepackage{amsmath,amssymb,amsfonts}
\usepackage{color}
\usepackage{graphicx}
\usepackage{rotating}
\usepackage{listings}
\usepackage{cite}
\usepackage{url}
\usepackage{latexsym}
\usepackage{makeidx}
\usepackage{color}
\usepackage{natbib}
\usepackage{array}
\usepackage{todonotes}

\lstset{
        basicstyle=\scriptsize\ttfamily, % Standardschrift
        numbers=left,               % Ort der Zeilennummern
        numberstyle=\tiny,          % Stil der Zeilennummern
        %stepnumber=2,               % Abstand zwischen den Zeilennummern
        numbersep=5pt,              % Abstand der Nummern zum Text
        tabsize=2,                  % Groesse von Tabs
        %extendedchars=true,         %
        breaklines=true,            % Zeilen werden Umgebrochen
        keywordstyle=\color{red}\ttfamily,
   		frame=b,         
%        keywordstyle=[1]\textbf,    % Stil der Keywords
 %       keywordstyle=[2]\textbf,    %
  %      keywordstyle=[3]\textbf,    %
   %     keywordstyle=[4]\textbf,   \sqrt{\sqrt{}} %
        stringstyle=\color{blue}\ttfamily, % Farbe der String
        showspaces=false,           % Leerzeichen anzeigen ?
        showtabs=false,             % Tabs anzeigen ?
        xleftmargin=17pt,
        framexleftmargin=17pt,
        framexrightmargin=5pt,
        framexbottommargin=4pt,
        %backgroundcolor=\color{lightgray},
        %showstringspaces=false,      % Leerzeichen in Strings anzeigen ?        
		language=Java
}
\lstloadlanguages{Java}
\usepackage{caption}
\DeclareCaptionFont{white}{\color{white}}
\DeclareCaptionFormat{listing}{\colorbox[cmyk]{0.88, 0.44, 0.0,0.0}{\parbox{\textwidth}{\hspace{15pt}#1#2#3}}  }
\captionsetup[lstlisting]{format=listing,labelfont=white,textfont=white, singlelinecheck=false, margin=0pt, font={bf,footnotesize}}

\definecolor{darkgreen}{cmyk}{0.7, 0, 1, 0.5}
\definecolor{darkblue}{rgb}{0.1, 0.1, 0.5}

\lstdefinelanguage{diff}
{
keywords={+, -, \ , @@, diff, index, new},
sensitive=false,
morecomment=[l][""]{\ },
morecomment=[l][\color{darkgreen}]{+},
morecomment=[l][\color{red}]{-},
morecomment=[l][\color{darkblue}]{@@},
morecomment=[l][\color{darkblue}]{diff},
morecomment=[l][\color{darkblue}]{index},
morecomment=[l][\color{darkblue}]{new},
morecomment=[l][\color{darkblue}]{similarity},
morecomment=[l][\color{darkblue}]{rename},
}

\makeindex


\usepackage{geometry,mflogo,xspace,texnames,path,booktabs,bm}
\usepackage[hyperindex,bookmarks,pdfborder=0,plainpages=false,pdfpagelabels]{hyperref}


%Settings applicable to the complete document

%Breaks URLs properly and draws them in a nice font
\urlstyle{sm}

\renewcommand{\ttdefault}{pcr} % Courier has a bold shape, default tt does not


%Decrease the default indentation of paragraphs
\parindent=0.3cm

%new or changed commands
\renewcommand\contentsname{Table of Content}
\newcommand{\chapref}[1]{Chapter~\ref{#1}}
\newcommand{\secref}[1]{Section~\ref{#1}}
\newcommand{\appref}[1]{Appendix~\ref{#1}}
\newcommand{\tabref}[1]{Table~\ref{#1}}
\newcommand{\figref}[2][]{Figure~\ref{#2}#1}
\newcommand{\listref}[2][]{Listing~\ref{#2}#1}

\begin{document}
\bibliographystyle{plainnat}

%Title Page
\pagestyle{empty}

\begin{titlepage}
\begin{center}
\begin{LARGE}
Norges teknisk-naturvitenskapelige universitet

Department of Computer and Information Science
\vspace{1.2in}

\end{LARGE}
\begin{Huge}
A Component based Game Architecture for Unknown Horizons\vspace{1.2in}
\vspace{1.7in}


\end{Huge}
\begin{LARGE}
Thomas Kinnen \vspace{0.6in}
\end{LARGE}

\begin{Large}
TDT4570 - Game Technology, Specialization Project
\vspace{1.2in}

\end{Large}
\end{center}

\end{titlepage}

%% Abstract
\chapter*{Abstract}
This is the abstract, done in the end.


%\input{acknowledgement.tex}


% Table of Contents
\setcounter{tocdepth}{1}                % Sets depth of table of contents. 0 is chapter, 1 is sections, 2 is subsections
\setcounter{secnumdepth}{2}             % Sets depth of numbering of toc contents
\tableofcontents

\pagestyle{headings}

% Chapters. Each one in its own file
% {{{=================== Introduction ======================


\section{Introduction}

\subsection{Motivation}
\UH{}\footnote{Unknown Horizons website: \url{http://www.unknown-horizons.org}} is an \OS{} real-time strategy game developed by a team of programmers, artists, game
designers and many more around the globe. The first revision was committed in late 2007\footnote{First commit to \UH{}:
\url{https://github.com/unknown-horizons/unknown-horizons/commit/53eec12fd8bb52ac1a6ccfdb097296c479499dfd}}.

As the project evolved the game's code architecture grew dynamically, without much planned structure or
designed architecture. This resulted in a very tight coupling between the different components inside the game, making
it difficult to add/change certain functionality in the game. This became clear when adding the boat builder building
a while back, which resulted in months of fixing introduced bugs.

\UH{} uses the outdated idea of making use of multiple inheritance to compose its in-game objects. Besides introducing
very tight coupling between the different classes the current approach also does not allow non programmers to add new
assets to the game. For an \OS{} project this is clearly not ideal, as user contributions would add great value to the
project and save valuable programming time.

The idea for this project is to research how this problem is solved in similar \OS{} games and to transfer the results to
the \UH{} source-code. 

\pagebreak

The following games have been chosen to be researched:
\begin{itemize}
    \item \BOW{}\footnote{Battle of Wesnoth website: \url{http://www.wesnoth.org}}
    \item \AD{}\footnote{0 A.D. website: \url{http://wildfiregames.com/0ad/}}
    \item \GLEST{}\footnote{Glest website: \url{http://megaglest.org/}}
\end{itemize}

\subsection{Problem Statement}
Three main questions should be answered by this project:
\begin{itemize}
    \item Which architecture do \OS{} games similar to \UH{} use to model their in-game objects?
    \item Can users add objects without modifying the game's code and if yes -- how?
    \item Can the \UH{} code-base be ported to a component-based architecture?
\end{itemize}

\subsection{Project Context}
This project is conducted for the course \textit{TDT4570 - Game Technology Specialization Project}\footnote{TDT4570 Project
description: \url{http://www.idi.ntnu.no/emner/tdt4570/}} which is part of
\textbf{NTNU}'s computer science master program.

%}}}



%\appendix

%\chapter{Patches}\label{appendix}


\chapter{Grammars}\label{appendixGrammars}


%\bibliography{references}

\end{document}
