\section{Conclusion and Future Work}
In this section I present a conclusion of this project and the future work that can follow it.

\subsection{Conclusion}



\subsection{Future Work}
On the research level it would be interesting to compare commercial games to the \OS{} systems described in this
report. It could provide a lot of insight of how far industry standards in gaming have made their way into the \OS{}
gaming world.

There is a lot of work remaining to finish the conversion from an inheritance based approach to the new component based
architecture. Many components have to be created, thus minimizing the hierarchy. A lot of time should be allocated for
this task, as even creating one component can take days of debugging and fixing to make the code run again. Because of
the tight coupling of the current code-base and the dynamic nature of Python refactoring is not easy and requires
thorough testing afterwords.

In order to ensure easy testing of newly created components the current collection of unit and systemtests should be
extended, to cover greater areas of the code. Especially system tests help with the refactoring as they test the
coordination between the components, which is likely to break when refactoring the code.

Currently the code uses a changelistener approach for class communication and to signal events. It could be an
interesting project to redesign this system and port it to a message based system, as described in \cite{Fh02ageneric}.
This could provide even looser coupling, making the component system more effective at modularizing the code-base.

