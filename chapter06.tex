\section{Conclusion and Future Work}
In this section I present a conclusion of this project and the future work that can follow it.

\subsection{Conclusion}
I set of with the goal of introducing a component based architecture into the \UH{} code-base. In order to complete the
task properly, I first looked at the few existing literature on the topic and conducted a case study on the four \OS{}
games \UH{} (to see where I am at), \BOW{}, \GLEST{} and \AD{}.

I found out that \UH{} does not have a component based system in-place yet, only a small start in terms of a
\textit{HealthComponent} and a \textit{StanceComponent} was made during the last \textit{Google Summer of Code}. I
examined an example class hierarchy to understand that the hierarchy is big with 16 involved classes and multiple
inheritance.

\BOW{} and \GLEST{} provide a semi-component based system. Objects are described using an extensible markup language
(WML and XML), making it easy for content contributors to add value to the game. Code wise both games don't follow a
pure component based architecture, but also use some form of inheritance.

\AD{} uses a purely component based architecture, representing an entity in code as an unique identifier to which a set
of components is mapped. It provides a powerful architecture which allows the developer to implement components in c++
or LUA, depending on the need for speed or simplicity. It also allows the exchange of all the LUA components in the form
of a mod, making the core c++ code more a game engine than base code for \AD{} specifically.

Based on this knowledge I designed a component system for \UH{} that can gradually be introduced into the existing
code-base. For this reason it is not as pure as the approach used in \AD{}, but it provides a lot more flexibility in
porting the existing code to the new system in small steps.

While implementing some basic components I quickly realized that porting the entire code base to such a new system in
one step would
be impossible. This work presents a solid foundation for further work on this topic, which I am planning to do in the
next months. It will help \UH{} become more content contributor and game-design friendly as it allows editing units and
buildings by editing simple to read and edit YAML files instead of editing SQLite databases.


\subsection{Future Work}
On the research level it would be interesting to compare commercial games to the \OS{} systems described in this
report. It could provide a lot of insight of how far industry standards in gaming have made their way into the \OS{}
gaming world.

There is a lot of work remaining to finish the conversion from an inheritance based approach to the new component based
architecture. Many components have to be created, thus minimizing the hierarchy. A lot of time should be allocated for
this task, as even creating one component can take days of debugging and fixing to make the code run again. Because of
the tight coupling of the current code-base and the dynamic nature of Python refactoring is not easy and requires
thorough testing afterwords.

In order to ensure easy testing of newly created components the current collection of unit and system-tests should be
extended, to cover greater areas of the code. Especially system tests help with the refactoring as they test the
coordination between the components, which is likely to break when refactoring the code.

Currently the code uses a changelistener approach for class communication and to signal events. It could be an
interesting project to redesign this system and port it to a message based system, as described in \cite{Fh02ageneric}.
This could provide even looser coupling, making the component system more effective at modularizing the code-base.

